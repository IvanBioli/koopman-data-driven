Throughout this work we will consider autonomous dynamical systems with finite-dimensional state space $\Omega\subseteq\R^d$ and discrete time steps according to a function $F$, i.e. dynamical systems whose evolution is characterized by the relation:
\begin{equation}
    \label{dynamical_system}
    \vb{x}_{n+1} = \vb{F}(\vb{x}_n)\,\,\,n\geq 0, \qquad \vb{F}:\Omega \to \Omega
\end{equation}
where $\vb{x}_0$ is a given initial condition. We are interested in analyzing the system's behavior from its trajectories $\vb{x}_0, \vb{x}_1, \vb{x}_2,...$, or more in general measurements of these trajectories collected through data. Observe that the discrete-time formulation is more natural when considering measurements, which are usually taken at discrete time steps. If we consider the autonomous continuous-time dynamical system described by
\begin{equation*}
    \frac{d}{dt}\vb{x}(t) = \vb{f}(\vb{x}(t)),
\end{equation*}
then sampling the trajectories with time steps $\Delta t$ we are back to the formulation in \eqref{dynamical_system} by considering
\begin{equation}
    \label{continuous_to_discrete}
    \vb{x}(t+\Delta t) = \vb{F}_{\Delta t}(\vb{x}(t)), \qquad \vb\vb{F}_{\Delta t}(\vb{x}(t)) = \vb{x}(t) + \int_{t}^{t+\Delta t}{\vb{f}(\vb{x}(\tau)) d\tau}.
\end{equation}

The classical approach to the study of dynamical systems is the geometric one, originated by the work of Poincaré \cite{henri_poincare_les_1899}. However, this geometric viewpoint, based on fixed points, local behaviors and invariant manifolds, is ill-suited to many real world applications \cite{budisic_applied_2012}. As pointed out in \cite{colbrook_rigorous_2021}, the fundamental challenges of the Poincaré's approach in a data-driven perspective are two: nonlinear dynamics, for which the geometric viewpoints provides good trajectories' approximations only locally and not for all initial conditions, and unknown dynamics. The latter case typically arises when the system's dynamic is too complex to describe analytically or partially unknown to , and we have only access to experimental data related to sequences of iterates sampled starting from different initial conditions. Koopman Operator theory \cite{koopman_dynamical_1932, koopman_hamiltonian_1931} provides an alternative viewpoint, which is a general framework for connecting data and measurements to the state space of a dynamical system \cite{arbabi_introduction_nodate}.

\section{Definition of the Koopman Operator}
When dealing with real world applications, especially when the dynamics is unknown, the state of the dynamical system is indirectly measured through data, which are variables related to the state of the system. We can mathematically formulate this assuming that data is evaluations of functions of the states. A function $g:\Omega\to\C$ used to indirectly measure the state of the dynamical system is called an \emph{observable}. 
Our goal is to be able to approximate as accurately as possible the dynamics of the system. To be able to do so from experimental data, i.e. from measurements, we need to understand how the observables evolve over time and here is where the Koopman Operator comes into play. The Koopman Operator advances the measurement forward in time, by one time step: it takes the state, advances the system by one time step and measures the system again. Given an observable $g:\Omega\to\C$, we define $[\mathcal{K}g](\vb{x}) = g(\vb{F}(\vb{x}))$. It is typical to define the Koopman Operator on (a subset of) the Hilbert space $L^2(\Omega, \omega)$, where $\omega$ is a positive measure on the state-space. Hence, given a suitable domain of observables $\mathcal{D}(\mathcal{K}) \subseteq L^2(\Omega, \omega)$ we define:
\begin{equation}
    \label{koopman_def}
    \begin{split}
       \mathcal{K} : \mathcal{D}(\mathcal{K}) &\longrightarrow L^2(\Omega, \omega)
       \\
       g & \longmapsto g \circ \vb{F}
    \end{split}    
\end{equation}

We can think of the Koopman Operator as lifting the dynamics from the state space to the space of observables. From this lifting we gain that, regardless of the linearity or nonlinearity of $\vb{F}$, the Koopman Operator is a linear operator and therefore to understand the dynamics of the system we can analyze the spectral properties of $\mathcal{K}$. However, the disadvantage is that the space of the observables is infinite dimensional and $\mathcal{K}$ can have a continuous spectrum. 

\section{Koopman eigenvalues, eigenfunctions and eigenmodes}

Since the definition of an eigenvalue-eigenfunction pair (abbr. \emph{eigenpair}) may slightly vary according to the context in which the Koopman operator theory is applied \cite{}, let us specify the following definition.
\begin{definition}[Koopman eigenpair]
Let $\phi_j:\Omega\to\Omega$ be an observable of the dynamical system and let $\lambda_j\in\C$. The pair $(\phi_j, \,\lambda_j)$ is called an eigenpair of the Koopman Operator $\mathcal{K}$ if $\mathcal{K}\phi_j = \lambda_j\phi_j$, i.e if $[\mathcal{K}\phi_j](\vb{x}) = \phi_j(\vb{F}(\vb{x})) = \lambda_j\phi_j(\vb{x})$.
\end{definition}

Observing that the composition of function is not only linear but also preserves the product, i.e. $(g_1 \cdot g_2) \circ f = (g_1 \circ f) \cdot (g_2 \circ f)$, it is straightforward that $\mathcal{K}(g_1\cdot g_2) = \mathcal{K}g_1\cdot \mathcal{K}g_2$. The following lemma is therefore trivial.
\begin{lemma}
\label{eigenpair_multiplication}
Let $(\phi_j, \,\lambda_j)$ and $(\phi_k, \,\lambda_k)$ be two not necessarily distinct eigenpairs of $\mathcal{K}$. Then also $(\phi_j\cdot\phi_k, \,\lambda_j\cdot\lambda_k)$ is an eigenpair of $\mathcal{K}$. In particular, given an eigenpair $(\phi, \,\lambda)$, also $(\phi^k, \,\lambda^k)$ is an eigenpair of $\mathcal{K}$ $\forall k\in\mathbb{N},\,\, k \geq 0$.
\end{lemma}

\subsection{Linear systems with simple spectrum}
Let us consider the case in which $\vb{F}(\vb{x}) = \vb*{A}\vb{x}$, $\vb*{A}\in\R^{d\times d}$, i.e. the linear case. Then the eigenvalues of $\vb*{A}$ are eigenvalues of $\mathcal{K}$ and the eigenvectors of $\vb*{A}$ are closely related to the associated eigenfunctions.

Let $\lambda$ be an eigenvector of $\vb*{A}$ and let $\vb{w}$ be an associated left eigenvector, i.e. an eigenvector of $\vb*{A}^*$ associated with $\overline{\lambda}$. If we define $\phi(\vb{x}) = \langle \vb{x},\vb{w} \rangle$, then
\begin{equation*}
    [\mathcal{K}\phi](\vb{x}) = \phi(\vb*{A}\vb{x}) = \langle \vb*{A}\vb{x},\vb{w} \rangle = \langle \vb{x}, \vb*{A}^*\vb{w} \rangle = \langle \vb{x}, \overline{\lambda}\vb{w} \rangle = \lambda \langle \vb{x},\vb{w} \rangle = \lambda \phi(\vb{x})
\end{equation*}
Hence $\lambda$ is an eigenvalue of $\mathcal{K}$, with corresponding eigenfunction $\phi$. Let us observe that, even in this very simple setting, from \Cref{eigenpair_multiplication} we obtain that if $\vb*{A}$ has at least one eigenvalue that is not a root of unity, then $\mathcal{K}$ has an infinite number of eigenvalues.

If in addition we make the assumption that $\vb*{A}$ is diagonalizable with a full set of eigenpairs $\{(\lambda_j, \vb{v}_j)\}_{j=1}^{d}$, it is possible to choose the corresponding eigenpairs $\{(\overline{\lambda}_j, \vb{w}_j)\}_{j=1}^{d}$ of $\vb*{A}^*$ such that $\langle \vb{v}_j, \vb{w}_k\rangle = \delta_{k,j}$. Therefore we can write for all $\vb{x}\in\R^d$:
\begin{equation}
    \label{decomposition_linear}
    \vb{x} = \sum_{j=1}^d \langle \vb{x}, \vb{w}_j\rangle \vb{v}_j = \sum_{j=1}^d \phi_j(\vb{x}) \vb{v}_j
\end{equation}
and the evolution of the system is
\begin{equation}
    \label{evolution_linear}
    \vb{F}(\vb{x}) = \vb*{A}\vb{x}  = \sum_{j=1}^d \phi_j(\vb{x}) \vb*{A}\vb{v}_j = \sum_{j=1}^d \lambda_j \phi_j(\vb{x}) \vb{v}_j = \sum_{j=1}^d [\mathcal{K}\phi_j](\vb{x})\vb{v}_j.
\end{equation}
The decomposition in \eqref{decomposition_linear} is nothing more than an expansion of $\vb{x}$ as a linear combination of the vectors $\vb{v}_j$, where the $\phi_j(\vb{x})$ are the coefficients. However, from the Koopman Operator's viewpoint it is a linear expansion of the (vector) full state observable as a linear combination of the eigenfunctions of $\mathcal{K}$, where now the role of the coefficients is played by the vectors $\vb{v}_j$ \cite{rowley_spectral_2009}. 

The vectors $\vb{v}_j$ are called \emph{Koopman modes} with respect to the observable $g$, which in this case is the full state (vector) observable $g(\vb{x}) = \vb{x}$. They are not associated with the Koopman Operator itself, but rather with the action of $\mathcal{K}$ on a particular observable. For instance if we consider another linear observable $g(\vb{x}) = \vb*{C}\vb{x}$ where $\vb*{C}\in\R^{m \times d}$ then we obtain
\begin{equation*}
	[\mathcal{K}g](\vb{x}) = g(\vb*{A}\vb{x})  = \sum_{j=1}^d \lambda_j \phi_j(\vb{x}) \vb*{C}\vb{v}_j.
\end{equation*}
i.e. $\{\vb*{C}\vb{v}_j\}_{j = 1}^d$ are the new Koopman modes.

We can therefore conclude that for linear dynamical systems with simple spectrum the two formulations, the one using $\mathcal{K}$ and the one using $\vb*{F}$ are strictly linked and almost equivalent from the computational point of view. In particular, the full state observable can be expressed as combination of eigenfunctions derived from the eigenvectors of $\vb*{A}^*$ and the Koopman modes coincide with the eigenvectors of $\vb*{A}$. This can be generalized to the case of non-simple spectrum, i.e. $\vb*{A}$ non diagonalizable, defining generalized eigenfunctions strictly linked with the generalized eigenvectors of $\vb*{A}$. More in general, the properties of the Koopman operator can be obtained in a quite straightforward fashion from the standard spectral analysis of $\vb*{A}$ and $\vb*{A}^*$. We point the reader to \cite{mezic_spectrum_2019} for a more comprehensive analysis.

\subsection{Koopman modes}
So far, the Koopman Operator theory applied to the linear case did not provide any new insight in the analysis of the dynamical systems. Let us now consider the more general non-linear case, in which the linearisation introduced by the Koopman operator (at the price of an infinite dimensional space) can significantly help us.

Let $g:\Omega\to\C^p$ be a vector valued observable. We can think of it as a vector gathering different measurements of the system state. Let us assume that each of its component $g_i$ lies in the closure of the Span of $J$ eigenfunctions of the Koopman Operator, where the case $J=+\infty$ is possible and often occurs. Then 
\begin{equation*}
	g_i = \sum_{j = 1}^J v_{ij}\phi_j, \qquad v_{ij}\in\C
\end{equation*}
and staking the weights into the vectors $\vb{v}_j = [v_{1j},\dots,v_{pj}]^T\in\C^p$ we can write
\begin{equation}
	g(\vb{x}) = \sum_{j = 1}^J \phi_j(\vb{x})\vb{v}_{j}.
\end{equation}
The vectors $\vb{v}_j$ are called the Koopman modes of the map $\vb{F}$ corresponding to the observable $g$. The evolution of the observable is then
\begin{equation}
	\label{evolution_nonlinear}
	g(\vb{x}_n) = [\mathcal{K}^ng](\vb{x}_0) = \sum_{j = 1}^J \lambda_j^n\phi_j(\vb{x}_0)\vb{v}_{j}.
\end{equation}
From \eqref{evolution_nonlinear} we can understand that the Koopman eigenvalue $\lambda_j$ characterizes the contribute of the corresponding Koopman mode $\vb{v}_j$ to the evolution of the observable over time: the phase of $\lambda_j$ determines its frequency, while the magnitude determines the growth rate.

As already discussed, the dynamical system defined by $\vb{F}$ and the one defined by $\mathcal{K}$ are two different ways of describing the same underlying phenomenon. The former is finite dimensional but in general non-linear, the latter is linear but infinite dimensional. The idea to link these two formulations is the full state observable $g(\vb{x}) = \vb{x}$ and the set $\{(\lambda_j, \phi_j, \vb{v}_j)\}_{j = 1}^J$ of $J$ tuples of Koopman eigenvalues, eigenfunctions and eigenmodes corresponding to the full state observable. Indeed, if the components of the full state observable lie in the Span of $\{(\lambda_j, \phi_j, \vb{v}_j)\}_{j = 1}^J$, the system evolution can be obtained either applying the complex and non-linear $\vb{F}$ to $\vb{x}$ or evolving $g$ linearly using $\mathcal{K}$ through
\begin{equation*}
	\vb{F}(\vb{x}) = [\mathcal{K}g](\vb{x}) = \sum_{j = 1}^J [\mathcal{K}\phi_j](\vb{x})\vb{v}_j = \sum_{j = 1}^J \lambda_j\phi_j(\vb{x})\vb{v}_j
\end{equation*}
with the behaviour of the system along each eigenfunction that is determined by the corresponding eigenvalue.