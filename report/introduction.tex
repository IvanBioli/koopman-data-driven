\section{Introduction to the problem and paper structure}
Dynamical systems are mathematical models describing the variability of a state over time, based on a fixed rule. The applications of dynamic systems are innumerable and range from classical mechanics, electrical circuits, fluid flows and thermodynamic systems to neuroscience, finance and epidemiology. Throughout this work we will consider autonomous dynamical systems with finite-dimensional state space $\Omega\subseteq\R^d$ and discrete time steps according to a function $F$, i.e. dynamical systems whose evolution is characterized by the relation:
\begin{equation}
    \label{dynamical_system}
    \vb{x}_{n+1} = \vb{F}(\vb{x}_n)\,\,\,n\geq 0, \qquad \vb{F}:\Omega \to \Omega
\end{equation}
where $\vb{x}_0$ is a given initial condition. We are interested in analyzing the system's behaviour from its trajectories $\vb{x}_0, \vb{x}_1, \vb{x}_2,...$, or, more in general, measurements of these trajectories collected through data. Observe that the discrete-time formulation is more natural when considering measurements, which are usually taken at discrete time steps. If we consider the autonomous continuous-time dynamical system described by
\begin{equation*}
    \frac{d}{dt}\vb{x}(t) = \vb{f}(\vb{x}(t)),
\end{equation*}
then sampling the trajectories with time steps $\Delta t$ we are back to the formulation in \eqref{dynamical_system} by considering
\begin{equation}
    \label{continuous_to_discrete}
    \vb{x}(t+\Delta t) = \vb{F}_{\Delta t}(\vb{x}(t)), \qquad \vb\vb{F}_{\Delta t}(\vb{x}(t)) = \vb{x}(t) + \int_{t}^{t+\Delta t}{\vb{f}(\vb{x}(\tau)) d\tau}.
\end{equation}

The classical approach to the study of dynamical systems is the geometric one, originated by the work of Poincaré \cite{henri_poincare_les_1899}. However, this geometric viewpoint, based on fixed points, local behaviours and invariant manifolds, is ill-suited to many real-world applications \cite{budisic_applied_2012}. As pointed out in \cite{colbrook_rigorous_2021}, the fundamental challenges of Poincaré's approach in a data-driven perspective are two: non-linear dynamics, for which the geometric viewpoints provides good trajectories approximations only locally and not for all initial conditions, and unknown dynamics. The latter case typically arises when the system's dynamic is too complex to be analytically described or partially unknown, and we have only access to experimental data related to sequences of iterates sampled starting from different initial conditions. Koopman Operator theory \cite{koopman_dynamical_1932, koopman_hamiltonian_1931} provides an alternative viewpoint, which is a general framework for connecting data and measurements to the state space of a dynamical system \cite{arbabi_introduction_2018}.
\subsection{Paper structure}
In this first chapter, we introduce the Koopman Operator and its spectral properties, analysing in details the case of linear dynamical systems. In \Cref{chapter_dmd} we introduce Dynamic Mode Decomposition (DMD), we link it to the Arnoldi method in the case of linear dynamical systems, and we present how it can be extended to the general case of nonlinear dynamics. In \Cref{chapter_edmd} we present Extended Dynamic Mode Decomposition (EDMD) for computing spectral properties of the Koopman Operator, discussing the spectral pollution problem and introducing the Residual Dynamic Mode Decomposition (ResDMD) algorithm for removing spectral pollution and approximating the pseudospectrum of the Koopman Operator. We also reproduce some of the numerical examples in \cite{colbrook_rigorous_2021}. Finally, in \Cref{chapter_kedmd} we present a kernel-based approach for dynamical systems with high dimensional data, analysing experimentally possible applications also to low dimensional data.

\section{Definition of the Koopman Operator}
When dealing with real-world applications, especially when the dynamics are unknown, the state of the dynamical system is indirectly measured through data, which are variables related to the system's state. We can mathematically formulate this fact by assuming that data are evaluations of functions of the states, leading to the following definition.

\begin{definition}[Observable]
A function $g:\Omega\to\C$ used to indirectly measure the state of the dynamical system is called an observable. 
\end{definition}

Our goal is to be able to approximate as accurately as possible the dynamics of the system. To achieve such goal from experimental data, i.e. from measurements, we need to understand how the observables evolve over time, and here is where the Koopman Operator comes into play. The Koopman Operator advances the measurement forward in time by one-time step: it takes the state, advances the system by one-time step and measures the system again. Given an observable $g:\Omega\to\C$, we define $[\mathcal{K}g](\vb{x}) = g(\vb{F}(\vb{x}))$. It is typical to define the Koopman Operator on (a subset of) the Hilbert space $L^2(\Omega, \omega)$, where $\omega$ is a positive measure on the state-space (in our examples we will use only the Lebesgue measure).

\begin{definition}[Koopman Operator]
Let $\Omega$ be the state-space of the dynamical system described by $\vb{F}:\Omega\to\Omega$ and let $\omega$ be a positive measure on the state-space. Given a suitable domain of observables $\mathcal{D}(\mathcal{K}) \subseteq L^2(\Omega, \omega)$ we define the Koopman Operator as:
\begin{equation}
    \label{koopman_def}
    \begin{split}
       \mathcal{K} : \mathcal{D}(\mathcal{K}) &\longrightarrow L^2(\Omega, \omega)
       \\
       g & \longmapsto g \circ \vb{F}
    \end{split}    
\end{equation} 
\end{definition}

We can think of the Koopman Operator as lifting the dynamics from the state space to the space of observables. From this lifting, we gain that, regardless of the linearity or nonlinearity of $\vb{F}$, the Koopman Operator is a linear operator. Therefore, to understand the dynamics of the system, we can analyze the spectral properties of $\mathcal{K}$. However, the disadvantage is that the space of the observables is infinite-dimensional and $\mathcal{K}$ can have a continuous spectrum. 

\section{Koopman eigenvalues, eigenfunctions and eigenmodes}

Since the definition of an eigenvalue-eigenfunction pair (abbr. \emph{eigenpair}) may slightly vary according to the context in which the Koopman Operator theory is applied, let us specify the following definition.
\begin{definition}[Koopman eigenpair]
Let $\phi_j:\Omega\to\Omega$ be an observable of the dynamical system and let $\lambda_j\in\C$. The pair $(\phi_j, \,\lambda_j)$ is called an eigenpair of the Koopman Operator $\mathcal{K}$ if $\mathcal{K}\phi_j = \lambda_j\phi_j$, i.e if $[\mathcal{K}\phi_j](\vb{x}) = \phi_j(\vb{F}(\vb{x})) = \lambda_j\phi_j(\vb{x})$.
\end{definition}

Observing that the composition of function is not only linear but also preserves the product, i.e. $(g_1 \cdot g_2) \circ f = (g_1 \circ f) \cdot (g_2 \circ f)$, it is straightforward that $\mathcal{K}(g_1\cdot g_2) = \mathcal{K}g_1\cdot \mathcal{K}g_2$. The following lemma is, therefore, trivial.
\begin{lemma}
\label{eigenpair_multiplication}
Let $(\phi_j, \,\lambda_j)$ and $(\phi_k, \,\lambda_k)$ be two not necessarily distinct eigenpairs of $\mathcal{K}$, such that the supports of $\phi_j$ and $\phi_k$ have an intersection with nonzero measure. Then also $(\phi_j\cdot\phi_k, \,\lambda_j\cdot\lambda_k)$ is an eigenpair of $\mathcal{K}$. In particular, given an eigenpair $(\phi, \,\lambda)$, also $(\phi^k, \,\lambda^k)$ is an eigenpair of $\mathcal{K}$ $\forall k\in\mathbb{N},\,\, k \geq 0$.
\end{lemma}

\subsection{Linear systems with simple spectrum}
Let us consider the case in which $\vb{F}(\vb{x}) = \vb*{A}\vb{x}$, $\vb*{A}\in\R^{d\times d}$, i.e. the linear case. Then the eigenvalues of $\vb*{A}$ are eigenvalues of $\mathcal{K}$ and the eigenvectors of $\vb*{A}$ are closely related to the associated eigenfunctions.

\begin{prop}
Let $\lambda$ be an eigenvector of $\vb*{A}$, let $\vb{w}$ be an associated left eigenvector, i.e. an eigenvector of $\vb*{A}^*$ associated with $\overline{\lambda}$, and let us define $\phi(\vb{x}) = \langle \vb{x},\vb{w} \rangle$. Then $\lambda$ is an eigenvalue of $\mathcal{K}$, with corresponding eigenfunction $\phi$.
\end{prop}
\begin{proof}
\begin{equation*}
    [\mathcal{K}\phi](\vb{x}) = \phi(\vb*{A}\vb{x}) = \langle \vb*{A}\vb{x},\vb{w} \rangle = \langle \vb{x}, \vb*{A}^*\vb{w} \rangle = \langle \vb{x}, \overline{\lambda}\vb{w} \rangle = \lambda \langle \vb{x},\vb{w} \rangle = \lambda \phi(\vb{x})
\end{equation*}
Hence $\lambda$ is an eigenvalue of $\mathcal{K}$, with corresponding eigenfunction $\phi$.
\end{proof}

Let us observe that, even in this very simple setting, from \Cref{eigenpair_multiplication} we obtain that if $\vb*{A}$ has at least one nonzero eigenvalue that is not a root of unity, then $\mathcal{K}$ has an infinite number of eigenvalues.

\begin{corollary}
Let us assume that $\vb*{A}\in\R^{d\times d}$ is diagonalizable with a full set of eigenpairs $\{(\lambda_j, \vb{v}_j)\}_{j=1}^{d}$. Let $\{(\overline{\lambda}_j, \vb{w}_j)\}_{j=1}^{d}$ be the eigenpairs of of $\vb*{A}^*$, with $\{\vb{w}_j)\}_{j=1}^{d}$ such that $\langle \vb{v}_j, \vb{w}_k\rangle = \delta_{k,j}$. Then we can rewrite $\vb{x}\in\R^d$:
\begin{equation}
    \label{decomposition_linear}
    \vb{x} = \sum_{j=1}^d \langle \vb{x}, \vb{w}_j\rangle \vb{v}_j = \sum_{j=1}^d \phi_j(\vb{x}) \vb{v}_j
\end{equation}
and the evolution of the system reads
\begin{equation}
    \label{evolution_linear}
    \vb{F}(\vb{x}) = \vb*{A}\vb{x}  = \sum_{j=1}^d \phi_j(\vb{x}) \vb*{A}\vb{v}_j = \sum_{j=1}^d \lambda_j \phi_j(\vb{x}) \vb{v}_j = \sum_{j=1}^d [\mathcal{K}\phi_j](\vb{x})\vb{v}_j.
\end{equation}
\end{corollary}

The decomposition in \eqref{decomposition_linear} is nothing more than an expansion of $\vb{x}$ as a linear combination of the vectors $\vb{v}_j$, where the $\phi_j(\vb{x})$ are the coefficients. However, from the Koopman Operator's viewpoint, it is a linear expansion of the (vector) full state observable as a linear combination of the eigenfunctions of $\mathcal{K}$, where now the role of the coefficients is played by the vectors $\vb{v}_j$ \cite{rowley_spectral_2009}. 

The vectors $\vb{v}_j$ are called \emph{Koopman modes} with respect to the observable $g$, which in this case is the full state (vector) observable $g(\vb{x}) = \vb{x}$. They are not associated to the Koopman Operator itself, but rather to the action of $\mathcal{K}$ on a particular observable. For instance, if we consider another linear observable $g(\vb{x}) = \vb*{C}\vb{x}$ where $\vb*{C}\in\R^{m \times d}$ then we obtain
\begin{equation*}
	[\mathcal{K}g](\vb{x}) = g(\vb*{A}\vb{x})  = \sum_{j=1}^d \lambda_j \phi_j(\vb{x}) \vb*{C}\vb{v}_j.
\end{equation*}
i.e. $\{\vb*{C}\vb{v}_j\}_{j = 1}^d$ are the new Koopman modes.

We can therefore conclude that for linear dynamical systems with simple spectrum, the two formulations, namely the one using $\mathcal{K}$ and the one using $\vb*{F}$, are strictly linked. In particular, the full state observable can be expressed as a combination of eigenfunctions derived from the eigenvectors of $\vb*{A}^*$, and the Koopman modes coincide with the eigenvectors of $\vb*{A}$. 

\subsection{Linear systems: general case}
The previous discussion can be generalized to the case of non-simple spectrum \cite{mezic_spectrum_2019}, i.e. $\vb*{A}$ non-diagonalizable, defining generalized eigenfunctions strictly linked to the generalized eigenvectors of $\vb*{A}$.
\begin{definition}
\label{gen_eigenfunction_def}
Let $\vb{A}\in\R^{d\times d}$ be a matrix with canonical Jordan form:
\begin{equation*}
    \vb*{J} = 
    \begin{bmatrix}
    \vb*{J}_{\lambda_1} & & & \\
    & \vb*{J}_{\lambda_2} & & \\
    & & \ddots & \\
    & & & \vb*{J}_{\lambda_s} \\
    \end{bmatrix}, \qquad 
    \vb*{J}_{\lambda_h} =
    \begin{bmatrix}
    \lambda_h & 1 & & & \\
    & \lambda_h & 1 & & \\
    & & \ddots &\ddots & \\
    & & & \lambda_h & 1\\
    & & & & \lambda_i \\
    \end{bmatrix} \in \C^{m_h}
\end{equation*}
and let $\vb{v}_h^1,\dots, \vb{v}_h^{m_h}$ be the generalized eigenvectors associated with the Jordan block $\vb*{J}_{\lambda_h}$, i.e.
\begin{equation*}
    \begin{split}
        \vb*{A}\vb{v}_h^1 &= \lambda_h \vb{v}_h^1 \\
        \vb*{A}\vb{v}_h^i &= \lambda_h \vb{v}_h^i + \vb{v}_h^{i-1} \qquad i = 2,\dots,m_h. 
    \end{split}
\end{equation*}
Let $\vb{w}_h^i$ be the dual basis vector to $\vb{v}_h^i$. We define the generalized eigenfunction as
\begin{equation}
    \phi_h^i(\vb{x}) = \langle \vb{x}, \vb{w}_h^i \rangle.
\end{equation}
\end{definition}

\begin{lemma}
Let us consider the generalized eigenfunctions as in \Cref{gen_eigenfunction_def}. Then:
\begin{equation*}
    \begin{split}
        [\mathcal{K}\phi_h^i](\vb{x}) &= \lambda_h \phi_h^i(\vb{x}) + \phi_h^{i+1}(\vb{x}) \qquad i = 1,\dots,m_h-1 \\
        [\mathcal{K}\phi_h^{m_h}](\vb{x}) &= \lambda_h \phi_h^{m_h}(\vb{x})
    \end{split}
\end{equation*}
\end{lemma}
\begin{proof}
The dual basis satisfies
\begin{equation*}
    \begin{split}
        \vb*{A}^*\vb{w}_h^i &= \overline{\lambda_h} \vb{w}_h^i + \vb{w}_h^{i+1} \qquad i = 1,\dots,m_h-1 \\
        \vb*{A}^*\vb{w}_h^{m_h} &= \overline{\lambda_h} \vb{w}_h^{m_h}
    \end{split}
\end{equation*}
thus it holds for $i = 1,\dots,m_h-1 $:
\begin{equation*}
    \begin{split}
        [\mathcal{K}\phi_h^i](\vb{x}) &= \phi_h^i(\vb*{A}\vb{x}) = \langle \vb*{A}\vb{x}, \vb{w}_h^i \rangle = \langle \vb{x}, \vb*{A}^*\vb{w}_h^i \rangle = \langle \vb{x}, \overline{\lambda_h} \vb{w}_h^i + \vb{w}_h^{i+1} \rangle = \\
        & =  \lambda_h \langle \vb{x}, \vb{w}_h^i \rangle + \langle \vb{x}, \vb{w}_h^{i+1} \rangle = \lambda_h \phi_h^i(\vb{x}) + \phi_h^{i+1}(\vb{x})
    \end{split}
\end{equation*}
and similarly $[\mathcal{K}\phi_h^{m_h}](\vb{x}) = \lambda_h \phi_h^{m_h}(\vb{x})$.
\end{proof}

Similarly to what has been done to the case of linear systems with simple spectrum, we can now recover the evolution of the full state observable and any linear observable from the above defined generalized eigenvalues
\begin{prop}
\label{prop_decomposition_jordan}
Let $\vb{g}$ be the full state observable, i.e. $\vb{g}(\vb{x}) = \vb{x}$. Then
\begin{equation*}
    \vb{g}(\vb{x}) = \vb{x} = \sum_{h = 1}^{s}\sum_{i = 1}^{m_h}\langle \vb{x}, \vb{w}_h^i\rangle\vb{v}_h^i = \sum_{h = 1}^{s}\sum_{i = 1}^{m_h} \phi_h^i(\vb{x}) \vb{v}_h^i
\end{equation*}
and the evolution of the system reads:
\begin{equation}
\begin{split}
    [\mathcal{K}\vb{g}](\vb{x}) &= \sum_{h = 1}^{s}\sum_{i = 1}^{m_h} [\mathcal{K}\phi_h^i](\vb{x}) \vb{v}_h^i = \sum_{h = 1}^{s} \left(\lambda_h\sum_{i = 1}^{m_h} \phi_h^i(\vb{x})\vb{v}_h^i + \sum_{i = 1}^{m_h-1}\phi_h^{i+1}(\vb{x})\vb{v}_h^i \right) =\\
    &= \sum_{h = 1}^{s} \left(\lambda_h P_h(\vb{x}) + D_h(\vb{x}) \right)
\end{split}
\end{equation}
where $P_h$ is the projection onto $\Span(\vb{v}_h^1,\dots,\vb{v}_h^{m_h})$ and $D_h$ is a nilpotent operator.
\end{prop}
\begin{proof}
We just need to show that $D_h$ is a nilpotent operator. It suffices to observe that $D_h$ is defined by
\begin{equation}
    D_h(\vb{x}) = D_h\left(\sum_{h = 1}^{s}\sum_{i = 1}^{m_h} \phi_h^i(\vb{x}) \vb{v}_h^i\right) = \sum_{i = 1}^{m_h-1}\phi_h^{i+1}(\vb{x})\vb{v}_h^i = \sum_{i = 2}^{m_h}\phi_h^{i}(\vb{x})\vb{v}_h^{i-1}
\end{equation}
hence by induction:
\begin{equation}
    D_h^k(\vb{x}) = \sum_{i = k+1}^{m_h}\phi_h^{i}(\vb{x})\vb{v}_h^{i-k}
\end{equation}
where we use the convention that the sum is equal to zero if $k+1>m_h$, i.e. if $k \geq m_h$.
\end{proof}

\begin{corollary}
Let $\vb*{A}\in\R^{d\times d}$ be a matrix and let $\mu_1, \dots, \mu_p$ be its eigenvalues, with algebraic multiplicity $\mathrm{am}_1, \dots, \mathrm{am}_p$. Let $\vb*{g}$ be the full state observable of the dynamical system described by $\vb*{A}$, then
\begin{equation*}
    [\mathcal{K}\vb{g}](\vb{x}) = \sum_{j = 1}^{p} \left(\mu_j P_j(\vb{x}) + D_j(\vb{x}) \right)
\end{equation*}
where $P_j$ is the projection onto the algebraic eigenspace associated to $\mu_j$, which is defined as the null space of $(\vb*{A} - \mu_j\vb*{I})^{\mathrm{ma}_j}$, and $D_j$ is a nilpotent operator of order equal to the maximum size of the Jordan blocks corresponding to the eigenvalue $\mu_j$. 
\end{corollary}
\begin{proof}
From \Cref{prop_decomposition_jordan}, supposing that $\mu_j = \lambda_{h_1} = \dots = \lambda_{h_r}$ and $\mu_j \neq \lambda_{h}$ for $h\notin\{h_1,\dots, h_r\}$, we can define:
\begin{equation*}
    \begin{split}
        P_j &= \sum_{i=1}^r P_{h_i}\\
        D_j &= \sum_{i=1}^r D_{h_i}.
    \end{split}
\end{equation*}
Then $P_j$ is the projection onto the algebraic eigenspace associated to $\mu_j$, and $D_{h_i}$ is a nilpotent operator with maximum order equal to the maximum size of the Jordan blocks corresponding to the eigenvalue $\mu_j$. Since $D_{h_i}D_{h_k} = 0$ if $i\neq k$, it holds
\begin{equation*}
    D_j^k = \sum_{i=1}^r D_{h_i}^k
\end{equation*}
and the thesis follows.
\end{proof}

Hence the action of the Koopman Operator on the full state observable can be obtained from the standard spectral analysis of $\vb*{A}$ and $\vb*{A}^*$, and the same is true for all linear observables $g(\vb{x}) = \langle \vb{c}, \vb{x} \rangle$.

\subsection{Koopman Mode Decomposition: nonlinear systems}
So far, the Koopman Operator theory applied to the linear case did not provide any new insight in the analysis of the dynamical systems. Let us now consider the more general non-linear case, in which the linearization introduced by the Koopman Operator (at the price of an infinite-dimensional space) can significantly help us.

We are interested in understanding the evolution of observables over time. This is particularly simple if the observable can be written as a linear combination of Koopman eigenfunctions. For this reason, the following definition was introduced in \cite{rowley_spectral_2009}.
\begin{definition}[Koopman Mode Decomposition]
Let $g:\Omega\to\C^p$ be a vector valued observable. Let us assume that each of its component $g_i$ lies in the closure of the Span of $J$ Koopman eigenfunctions, where the case $J=+\infty$ is possible (and often occurs). Then we can write 
\begin{equation*}
	g_i = \sum_{j = 1}^J v_{ij}\phi_j, \qquad v_{ij}\in\C
\end{equation*}
and staking the weights into the vectors $\vb{v}_j = [v_{1j},\dots,v_{pj}]^T\in\C^p$ 
\begin{equation}
    \label{koopman_modes}
	g(\vb{x}) = \sum_{j = 1}^J \phi_j(\vb{x})\vb{v}_{j}.
\end{equation}
This type of decomposition based on the Koopman eigenfunctions is named Koopman Mode Decomposition (KMD). The vectors $\vb{v}_j$ are called the Koopman modes of the map $\vb{F}$ corresponding to the observable $g$. 
\end{definition}

We can think of a vector valued observable $\vb{g}$ as a vector gathering different measurements of the system state. By linearity of the Koopman operator, the evolution of the observable is
\begin{equation}
	\label{evolution_nonlinear}
	g(\vb{x}_n) = [\mathcal{K}^ng](\vb{x}_0) = \sum_{j = 1}^J \lambda_j^n\phi_j(\vb{x}_0)\vb{v}_{j}.
\end{equation}
From \eqref{evolution_nonlinear} we can understand that the Koopman eigenvalue $\lambda_j$ characterizes the contribution of the corresponding Koopman mode $\vb{v}_j$ to the evolution of the observable over time: the phase of $\lambda_j$ determines its frequency, while the magnitude determines the growth rate.

As already discussed, the dynamical system defined by $\vb{F}$ and the one defined by $\mathcal{K}$ are two different ways of describing the same underlying phenomenon. The former is finite dimensional but in general non-linear, the latter is linear but infinite dimensional. The idea to link these two formulations is the full state observable $g(\vb{x}) = \vb{x}$ and the set $\{(\lambda_j, \phi_j, \vb{v}_j)\}_{j = 1}^J$ of $J$ tuples of Koopman eigenvalues, eigenfunctions and eigenmodes corresponding to the full state observable. Indeed, if the components of the full state observable lie in the Span of $\{(\lambda_j, \phi_j, \vb{v}_j)\}_{j = 1}^J$, the system evolution can be obtained either applying the complex and non-linear $\vb{F}$ to $\vb{x}$ or evolving $g$ linearly using $\mathcal{K}$ through
\begin{equation*}
	\vb{F}(\vb{x}) = [\mathcal{K}g](\vb{x}) = \sum_{j = 1}^J [\mathcal{K}\phi_j](\vb{x})\vb{v}_j = \sum_{j = 1}^J \lambda_j\phi_j(\vb{x})\vb{v}_j
\end{equation*}
with the behaviour of the system along each eigenfunction that is determined by the corresponding eigenvalue.