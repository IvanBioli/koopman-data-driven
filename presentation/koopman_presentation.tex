\documentclass{beamer}
\usetheme{Boadilla}
\title[Semester Project]{Rigorous data-driven computation of spectral properties of Koopman operators for dynamical systems}
\author[Ivan Bioli]{\emph{Author}: Ivan Bioli\\[5mm]{\emph{Professor}: Daniel Kressner \\ \emph{Supervisor}: Alice Cortinovis}}
\institute[EPFL]{École Polytechnique Fédérale de Lausanne\\Master in Computational Sciences and Engineering}
\date{May 30, 2022}
\usepackage{preamble_presentation}


\begin{document}
\begin{frame}
\centering
\titlepage
\end{frame}

\begin{frame}[fragile]{The Koopman Operator}
Autonomous dynamical systems with finite-dimensional state space and discrete time steps:
\begin{equation}
    \vb{x}_{n+1} = \vb{F}(\vb{x}_n)\,\,\,n\geq 0, \qquad \vb{F}:\Omega \to \Omega
\end{equation}
\begin{definition}[Observable]
A function $g:\Omega\to\C$ used to indirectly measure the state of the dynamical system is called an observable. 
\end{definition}
\begin{definition}[Koopman Operator]
Given a suitable domain of observables $\mathcal{D}(\mathcal{K}) \subseteq L^2(\Omega)$ we define the Koopman Operator as:
\begin{equation}
    \label{koopman_def}
    \begin{split}
       \mathcal{K} : \mathcal{D}(\mathcal{K}) &\longrightarrow L^2(\Omega, \omega)
       \\
       g & \longmapsto g \circ \vb{F}
    \end{split}    
\end{equation} 
\end{definition}
\end{frame}

\begin{frame}[fragile]{Linear dynamical systems}
\centering
$\vb{F}(\vb{x}) = A\vb{x}$, $\vb{A}\in\R^{d\times d}$
\begin{prop}
Let us assume that $\vb{A}\in\R^{d\times d}$ is diagonalizable with a full set of eigenpairs $\{(\lambda_j, \vb{v}_j)\}_{j=1}^{d}$. Let $\{(\overline{\lambda}_j, \vb{w}_j)\}_{j=1}^{d}$ be the eigenpairs of of $\vb{A}^*$, with $\{\vb{w}_j)\}_{j=1}^{d}$ such that $\langle \vb{v}_j, \vb{w}_k\rangle = \delta_{k,j}$. Then we can rewrite $\vb{x}\in\R^d$:
\begin{equation}
    \label{decomposition_linear}
    \vb{x} = \sum_{j=1}^d \langle \vb{x}, \vb{w}_j\rangle \vb{v}_j = \sum_{j=1}^d \phi_j(\vb{x}) \vb{v}_j.
\end{equation}
It holds $[\mathcal{K}\phi_j](\vb{x}) = \lambda_j\phi_j$ and the evolution of the system reads
\begin{equation}
    \label{evolution_linear}
    \vb{F}(\vb{x}) = \vb{A}\vb{x}  = \sum_{j=1}^d \phi_j(\vb{x}) \vb{A}\vb{v}_j = \sum_{j=1}^d \lambda_j \phi_j(\vb{x}) \vb{v}_j = \sum_{j=1}^d [\mathcal{K}\phi_j](\vb{x})\vb{v}_j.
\end{equation}
\end{prop}
\end{frame}

\begin{frame}[fragile]{Nonlinear dynamical systems}
\begin{definition}[Koopman Mode Decomposition]
$g:\Omega\to\C^p$ a vector valued observable s.t. each of its component $g_i$ lies in the closure of the Span of $J$ Koopman eigenfunctions, where the case $J=+\infty$ is possible (and often occurs). Then we can write $\displaystyle{g_i = \sum_{j = 1}^J v_{ij}\phi_j, \, v_{ij}\in\C}$ and staking the weights into the vectors 
%$\vb{v}_j = [v_{1j},\dots,v_{pj}]^T\in\C^p$ 
\begin{equation}
    \label{koopman_modes}
	g(\vb{x}) = \sum_{j = 1}^J \phi_j(\vb{x})\vb{v}_{j}.
\end{equation}
\end{definition}
\begin{itemize}
    \item $\vb{F}(\vb{x}) = [\mathcal{K}g](\vb{x}) = \sum_{j = 1}^J [\mathcal{K}\phi_j](\vb{x})\vb{v}_j = \sum_{j = 1}^J \lambda_j\phi_j(\vb{x})\vb{v}_j$
    \item $g(\vb{x}_n) = [\mathcal{K}^ng](\vb{x}_0) = \sum_{j = 1}^J \lambda_j^n\phi_j(\vb{x}_0)\vb{v}_{j}$ 
    \item Idea: link the two formulations via the full state observable $g(x) = x$
\end{itemize}
\end{frame}

\begin{frame}[fragile]{Dynamic Mode Decomposition (DMD)}
Linear dynamical system: $x_{n+1} = Ax_n$. Access to a sequence of snapshots:
\begin{equation*}
    \vb{V}_1^N = \left[\vb{v}_1, \vb{v}_2, \dots, \vb{v}_N\right] = \left[\vb{v}_1, \vb{A}\vb{v}_1, \dots, \vb{A}^N\vb{v}_1\right]
\end{equation*}
\alert{\textbf{Idea:}} write $\vb{v}_N = a_1 \vb{v}_1 + \dots + a_{N-1} \vb{v}_{N-1} + \vb{r}$ and minimize the residual.
\begin{block}{DMD (Arnoldi-based version)}
\begin{equation*}
    \vb{A}\vb{V}_1^{N-1}  = \vb{V}_1^{N-1}\vb{S} + \vb{r}\vb{e}_{N-1}^T, \quad 
    \vb{S} :=
   \begin{bmatrix}
   0     &        &       &      & a_1 \\
   1     & 0      &       &      & a_2 \\
         & \ddots & \ddots&      & \vdots \\ 
         &        & 1     & 0    & a_{N-2} \\
         &        &       & 1    & a_{N-1} \\
   \end{bmatrix}.
\end{equation*}
To minimize the norm of the residual $\vb{V}_1^{N-1} = \vb{Q}\vb{R}$ and:
\begin{equation*}
    \vb{a} = \vb{R}^{-1} \vb{Q}^*\vb{v}_N = (\vb{V}_1^{N-1})^{\dagger}\vb{v}_N.
\end{equation*}
\end{block}
\end{frame}

\begin{frame}{DMD and the Arnoldi method}
    
\end{frame}

\begin{frame}{Main references}
\nocite{*}
\printbibliography
\end{frame}

\end{document}